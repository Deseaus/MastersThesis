% Chapter 1

\chapter{Introduction} % Chapter title
\label{ch:introduction} % For referencing the chapter elsewhere, use \autoref{ch:introduction} 

%----------------------------------------------------------------------------------------

\noindent Translation is the shared object of study of two research communities: \ac{TS} and \ac{MT}, which is part of the larger \ac{NLP} community. Despite this common interest, it comes as both a surprise and a concern that there is little to no interaction between these communities and their research. Each one approaches translation from a different perspective, using different methodological tools and with different aims. The most evident difference is on the definition of the object of study itself: translation. While in \ac{TS} many a pages and thoughts have been put into trying to define what translation is and how both translators as subjects and translating as a process fit together to create the final product, in \ac{MT} all this is taken for granted and nearly no thought is spared as to what it means to translate.

\acf{CAT}, where translators use computers and specialised software to help them translate, seems a natural meeting point for these two communities. Indeed, in the early days of \ac{MT} research, \ac{CAT} was proposed as the way to go in the seminal \spacedlowsmallcaps{ALPAC} report \parencite{ALPAC}, again years later by \textcite{kay1980proper} and instisted upon yet again by \textcite{kay1997still}. Despite these influential papers, only recently has strong research into \ac{CAT} started picking up. Up until now, the field of \ac{CAT} has seen the introduction of some important technologies such as the use of \ac{TM} and glossaries. Apart from these two technologies, commercial \ac{CAT} tools have mainly focused on the project management and team coordination aspects of translating.

With the recent improvement in perceived quality of \ac{MT} output, especially after the introduction of the Moses \acs{SMT} system \parencite{koehn2007moses}, \ac{CAT} now seems attractive to \ac{MT} researchers. The focus, however, has been on \acf{PE}. In \ac{PE}, the translator is charged with nothing but ``cleaning up'' a machine translation to bring it up to acceptable quality. The story goes that this approach is faster and sufficient for certain texts that don't need full human translation quality, but this is not always the case. Humans are treated as simple quality assurance mechanisms instead of as people with a certain set of skills that could be put to good use. The move to \ac{PE} seems to serve more the interests of claiming \ac{MT} is viable and economical than the interests of improving translator's working conditions.

I argue that the problems with \ac{PE} and with \ac{TM} before it --- and more generally of translation carried out between humans and computers --- arise from the compartmentalised approach to \ac{CAT} tool research and development. \ac{NLP} researchers approach the problem from their knowledge and skills, often forgetting that the systems they develop have to work well with people, a very different problem to algorithm development. Researchers can end up creating tools that not only interact poorly with translators, but also potentially work against them. Similarly, translators and researchers from \ac{TS} are also weary of the claims that a machine can do what human translators can, to the point of taking over their jobs or a considerable portion of them. To this end, numerous studies (see \autoref{ch:cat}) have tried to surface all the shortcomings of these approaches. 

This situation is not desirable for either community. What is needed is to integrate both of them and collaborate in developing \ac{CAT} tools. The basic step is to change development methodologies and move away from the current state where one community creates a tool and puts it out there, and the other has to evaluate it and pick holes in it.

This thesis proposes a research methodology (\autoref{ch:methodology}) for developing \ac{CAT} tools with a tighter-knit integration of the skills of both communities. The gist is that theoretical research in \ac{TS} and translator surveys should uncover needs which can potentially be addressed with \ac{NLP} technologies. \ac{NLP} experts can then develop a prototype technology which solves this problem. It does not necessarily have to be fully-fledged, but it should have the potential to be so. Finally, \ac{TPR} methodologies can be used to prove that this new technology actually affects translations and translators positively. If all goes well, a full implementation can be left to future research projects in academia or in commercial applications.

My hypothesis is that \ac{CAT} tools developed on the basis of the proposed methodology should generate sufficient data to allow the tool to be evaluated instrinsically and extrinsically. In order to evaluate the methodology, in this thesis I use it to develop and evaluate StyleCheck, a tool that implements style guides. The results (\autoref{ch:results}) of the tool evaluation are presented, and a metaevaluation of the methodology is discussed.

It is hoped that through the use of this methodology, \ac{CAT} tools developed in the future will be liked by translators and have a positive impact on the work they carry out. In a nutshell, \ac{CAT} at the service of translators.
