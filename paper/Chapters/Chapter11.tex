\chapter{Conclusion}

\label{ch:conclusion}

\noindent This thesis has presented a unified methodology for developing and evaluating \ac{CAT} tools. The methodology has then been used to develop StyleCheck, a tool that helps translators by giving them hints about style guide rules that should be applied while they are translating.

%----------------------------------------------------------------------------------------

\section{StyleCheck}

\noindent Results show StyleCheck acheived its main goal: leading to a higher rate of rule application. Using it lead to more style rules being applied when compared with translation from scratch and post-editing.

StyleCheck does not seem to burden the translation process. Translators prefer it to post-editing. Speed is comparable to from-scratch translation, but the \spacedlowsmallcaps{UI} needs some work. Although not considered distracting nor intrusive, the interaction can be improved upon, especially with regards to adapting and rewriting the style guide rules to make them terser.

Despite the advantages, developing the rules themselves requires a lot of manual work. Inventoring all the options that a rule should match can be tricky, and some instances can fall through the net. 

%----------------------------------------------------------------------------------------

\section{Methodology}

\noindent The \ac{CAT} tool development methodology described in this thesis proved its worth. A simple survey among translators came up with plenty of suggestions for \ac{CAT} tool improvements, as did a quick read through translation theory.

Methods developed in \ac{TPR}, when used in combination (triangulation), managed to generate a wealth of data into many aspects of a tool: how translators use it, how it affects the way they translate and the impact on the final translation. This was possible even considering the only the most basic \ac{TPR} methods were used; more advanced methods such as eyetracking and keylogging should further improve understanding of how a tool is used.

As for specific findings related to evaluation, it has been shown that in questionnaires it is best to avoid questions related to ease/difficulty or like/dislike, as the text itself can weigh more heavily than the setup used. It is thus recommended to use questions that directly ask if translators would prefer to translate without a certain tool.

Continuing on the topic of evaluation, time as a metric should be used with care. Other studies that claim speed improvements in some setups or using certain tools fail to consider translation as a whole and ignore time-consuming tasks. Data presented in this thesis showed that post-editing speed increases only apply to the light variant with minimal changes. Full post-editing, where it is used as a \ac{CAT} tool, doesn't show large speed-ups and can even slow down some translators.

Prototyping as used in this thesis to develop StyleCheck is shown to be very useful as a feedback loop. A relatively modest investment in time and effort to build a prototype still manages to generate sufficient data about how the full tool will perform. Prototyping also allowed for problems with the tool (such as the hints being too long and always-on) to surface at an early stage so they can be quickly fixed.

All in all, work carried out in this thesis provides solid ground for the \ac{CAT} tool development methodology to be adopted and used in future research.
