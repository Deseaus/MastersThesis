% Chapter 8

\chapter{Questionnaires} % Chapter title
\label{ch:questionnaires} % For referencing the chapter elsewhere, use \autoref{ch:name} 

%----------------------------------------------------------------------------------------

\graffito{Note: For screenshots of the actual questionnaire, please see \autoref{an:questionnaires}.}

\noindent For this study, two questionnaires were created. The initial questionnaire was completed by participants before carrying out the translation tasks, while the final questionnaire was filled in after.

The questionnaires contained both open and closed questions. An overview of the questions and the research objectives each one aimed to cover are provided in the following subsections. Each question is given a number, although the actual questionnaire did not number the questions so that it didn't seem too long to participants. For this same reason, each questionnaire was presented split into pages and a progress bar shown at the bottom of each page. Options for the closed questions were randomly shuffled. 

Given that the majority of participants were likely to be native Spanish speakers and non-native English speakers, the questions were written in English but participants could answer the longer questions in either Spanish or English. Since they would be translating from English, a certain level of knowledge of English was presupposed and it wasn't deemed necessary to offer the questions also in Spanish.


%------------------------------------------------

\section{Initial Questionnaire}

\noindent This questionnaire gathered basic demographic data on the participants, as well as their previous experience with translation and various translation-related tasks. It was also exploratory in nature in that it tried to discover what tasks translators would like to be automated, as well as their opinions on \ac{MT} and it's relation to \ac{CAT}.

The specific research aims for the initial questionnaire and the questions related to them are the following:

\graffito{Note: numbers in brackets refer to the specific questions in the initial questionnaire. Please see \autoref{an:questionnaires} for the full questions.}
\begin{itemize}
\item Basic participant demographics (age, gender, occupation, languages) [Questions 2, 3, 4, 5, 6]
\item Participant's experience and training as translators [Questions 7, 8, 9, 10]
\item Participant's experience with specific translation-related tasks [Question 11]
\item Participant's experience with specific translation-related tools [Questions 18, 19, 20]
\item What do translators themselves consider translation to be? [Questions 1, 15, 16, 17]
\item What are the most important considerations when translating? [Questions 15, 16, 17]
\item What tasks carried out during translation are candidates for automation? [Questions 12, 13, 14]
\item What attitudes do translators have towards machine translation? [Questions 21, 22, 23]
\end{itemize}

%------------------------------------------------

\section{Final Questionnaire}

\noindent The final questionnaire asked translators to compare the three translation tasks performed and asked specific details about each individual task. Following \textcite{krings2005wege}, this questionnaire can be categorised as a retrospective questionnaire (related to a specific task) and as online (the researcher was not present while the questions were answered). 

\textcite{christensen2011studies} notes that a downside to retrospective questionnaires is the risk that subjects might make up explanations since they have to retrieve information from long-term memory. This study tried to partially offset this concern by providing participants with the full \ac{ST} (and \ac{MT} suggestion for the second task) directly above the final questionnaire, so that they could more easily remember the texts and answer the questions (see \autoref{fig:web_final}). The participant's own translations were not provided so that they would not get frustrated at seeing any possible mistakes they had made. 

The specific research aims for the final questionnaire and the questions related to them are described in the following list:

\graffito{Note: numbers in brackets refer to the specific questions in the final questionnaire. Please see \autoref{an:questionnaires} for the full questions.}
\begin{itemize}
\item Which setup was preferred by the translators? [Questions 14, 15, 22, 23, 32, 33]
\item Which factors made a text easy or difficult to translate? [Question 3, 4, 5, 6]
\item Which setup helped translators produce a better translation (perceived or real)? [Question 15, 23, 33]
\item Did they perceive any differences regarding how in control they were of their own final translations between the setups? [Questions 7, 8, 9, 10]
\item Did their perceived translation speed vary significantly from their real translation speed? [Questions 11, 17, 25]
\item Were the style hints useful for participants? [Question 28, 29, 30, 31, 32, 33]
\item Did participants find the \ac{MT} suggestions useful? [Question 20, 21, 22]
\end{itemize}

Some of the questions were exploratory in nature and could be categorised into various of the previous aims depending on the participant's answer. Very general and open questions were asked to encourage participants to come up with their own ideas about the three setups. Mainly, these questions were 3, 4, 5, 6, 12, 13, 16, 18, 19, 24, 26, 27, 34 and 35.

Questions 14, 20, 22 and 32 followed those used by \textcite{carl2015post}. The wording was slightly adapted to fit in with the current study. The question ``Overall, how satisfied were you with the task of translating Text X?'' (questions 14, 22 and 23) was complemented with ``Overall, how satisfied were you with the quality of your final translation of Text X'' (questions 15, 23 and 33) to separate the performance of the actual task from its output. The single question offered by \textcite{carl2015post} could be ambiguous and make participants hesitate between liking the task but not being satisfied with its final result.

A possibility that was considered was to include the System Usability Scale (\spacedlowsmallcaps{SUS}) \parencite{brooke1996sus} questions into the final questionnaire. The \spacedlowsmallcaps{SUS} is a widely-used set of 10 questions that measure the usability of a system. Despite the advantages of using a well-known questionnaire, it was decided to not include them. The 10 questions would have had to be provided for each of the three setups to allow for comparison. An extra 30 questions added to the final questionnaire was deemed to make it too long and fatiguing for participants.


