% Appendix X

\chapter{Wikipedia Texts}
\label{app:texts}

%----------------------------------------------------------------------------------------

% Content begins here

\section{Source Texts}

\noindent The following subsections contain the Wikipedia texts that participants were asked to translate. All links to other Wikipedia articles were removed and not shown to participants, as it was considered too much work for this experiment to ensure that the translation contained the correct links in Spanish.

\subsection{History of Artificial Intelligence}

\begin{quotation}
\noindent The field of AI research was founded at a conference on the campus of Dartmouth College in the summer of 1956. 

[\dots]

In 1973 [\dots] the U.S. and British Governments stopped funding undirected research into artificial intelligence. Seven years later, a visionary initiative by the Japanese Government inspired governments and industry to provide AI with billions of dollars, but by the late 80s the investors became disillusioned and withdrew funding again. This cycle of boom and bust, of "AI winters" and summers, continues to haunt the field.
\flushright\parencite{wikihistory}
\end{quotation}

\subsection{Charlotte Gyllenhammer}

\noindent A slight motification was introduced to this text in order to include a style error. The Wikipedia Spanish Style Guide \parencite{wikistyle} states that lists which can be ordered chronologically, should be ordered in this fashion. Although the original text correctly ordered the artist's works (``\textit{Belle}, 1998, \textit{Disobedience}, 1998, \textit{Fall}, 1999, and more recently \textit{Hang} 2006''), the order of \textit{Disobedience} and \textit{Fall} was swapped, making a clearly unordered list. Also, the bold font styling of ``\textbf{Charlotte Gyllenhammer}'' was removed as it was deemed there were sufficient other style guide considerations in the text.

\begin{quotation}
\noindent Charlotte Gyllenhammar, born 1963, is a fine artist based in Stockholm, Sweden. She began her career as a painter, but swiftly moved on to sculpture and installation after completing her studies at the Royal College of Art in London. 

[\dots] 

The work entitled \textit{Die for You} was the first step in a progression of images and environments that invert perspective. For example, confinement and inversion are evident in her video/photographic series of suspended women entitled \textit{Belle}, 1998, \textit{Fall}, 1999,\textit{Disobedience}, 1998, and more recently \textit{Hang} 2006.
\flushright\parencite{wikicharlotte}
\end{quotation}

\subsection{Garfield}

\noindent The first sentence of the text was shortened from the original, since the long length of the paragraph slowed down the StyleCheck tool and could even block it. The original read ``[\dots] any form of exertion; his constant shedding (which constantly annoys Jon); and his abuse of Odie and Jon as well as his obsession with mailing Nermal to Abu Dhabi.''


\begin{quotation}
\noindent Many of the gags focus on Garfield's obsessive eating and obesity; his fear of spiders (many of these can be found in the 7th strip comic collection type book ); his hate of Mondays, diets, and any form of exertion; and his obsession with mailing Nermal to Abu Dhabi.  Though he will eat nearly anything (with the exception of raisins and spinach), Garfield is particularly fond of lasagna; he also enjoys eating Jon's houseplants and other pets (mainly birds and fish).
\flushright\parencite{wikigarfield}
\end{quotation}

\section{MT Translation Suggestions}

\noindent Machine translation suggestions were obtained by running the text fragments through Google Translate\footnote{\url{https://translate.google.com/}, translations were carried out on 4 May 2015.}. Below are the results as they were used in the \ac{PE} setup of the experiments.

\subsection{History of Artificial Intelligence (MT)}

\begin{quotation}
\noindent El campo de la investigación en IA fue fundada en una conferencia en el campus de la universidad de Dartmouth en el verano de 1956.

En 1973 los EE.UU. y los gobiernos británicos detuvimos financiación de la investigación no dirigida en inteligencia artificial. Siete años más tarde, una iniciativa visionaria de los gobiernos Gobierno inspirado japonés y la industria para proporcionar AI con miles de millones de dólares, pero a finales de los años 80 los inversores se desilusionó y se retiró la financiación de nuevo. Este ciclo de auge y caída, de los "inviernos y veranos AI", sigue acosando el campo.
\end{quotation}

\subsection{Charlotte Gyllenhammer (MT)}

\begin{quotation}
\noindent Charlotte Gyllenhammar, nacido en 1963, es un artista plástico con sede en Estocolmo, Suecia. Comenzó su carrera como pintor, pero rápidamente se trasladó a la escultura y la instalación después de completar sus estudios en el Royal College of Art de Londres.

El trabajo titulado Die for You fue el primer paso en una progresión de imágenes y ambientes que invierten perspectiva. Por ejemplo, el parto y la inversión son evidentes en su video / serie fotográfica de las mujeres en suspensión con derecho Belle, 1998, Otoño, 1999, Desobediencia, 1998, y más recientemente Cuelgue 2006.
\end{quotation}

\subsection{Garfield (MT)}

\begin{quotation}
\noindent Muchos de los gags se centran en la alimentación y la obesidad obsesivo de Garfield; su miedo a las arañas (muchos de ellos se puede encontrar en el séptimo tira tipo de colección cómica del libro); su odio de los lunes, dietas, y cualquier forma de ejercicio; y su obsesión por correo Nermal a Abu Dhabi. A pesar de que va a comer casi cualquier cosa (con la excepción de las pasas y las espinacas), Garfield es particularmente aficionado a la lasaña; también le gusta comer las plantas de interior de Jon y otros animales (principalmente aves y peces).
\end{quotation}

