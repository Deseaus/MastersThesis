% Chapter 4

\chapter{CAT Research Methodology} % Chapter title
\label{ch:methodology}

\noindent Developments in the field of \ac{CAT} such as those mentioned in the previous chapter have been created without an extensive study into the potential effects they have on translators. While reuse, speed and ---~increasingly~--- reduced cognitive effort are claimed as advantages of each new improvement, studies have brought to light the not-so-obvious influences these tools have on the translation process, on translators and on translations themselves. Given that \ac{TM} and \ac{CAT} tools in general are now an essential part of a translator's environment \parencite{o2012translation}, it is probably time to think through what approach to take when developing further improvements.

A common theme of the \ac{CAT} tools discussed in the previous chapter is that they weigh down human translation with their own inherent limitations and simplifications. As seen in \autoref{sec:mt_theory}, the fact that \ac{MT} is basically stuck in the equivalence paradigm can be seen as both a limitation and simplification: trying to include all the external variables into the models used by \ac{SMT} systems would most likely make the problem intractable. \textcite{hardmeier2014discourse} elaborates on translation theories and some of the simplifications \ac{SMT} adopted and still adopts to solve the translation problem computationally. \textcite{hardmeier2014discourse} discusses two notable examples: the existence of some kind of equivalence between the \ac{ST} and the \ac{TT}, which enables the notion of word alignments between a \ac{ST} and its \ac{TT}; and translating only one sentence at a time without considering the wider textual context. These design limitations are now burdened onto translators through \ac{PE} and \ac{TM}s (even more so now that research into unifying \ac{TM}s and \ac{MT} is picking up).

It is of no surprise that translators tend to dislike using these tools, given that they make their work mechanical and less creative \parencite{christensen2011impact}. Taking a finer-grained look at \ac{PE}, the very term post-edit seems to convey the idea that the hard work is already done by the system, and just some cosmetic changes are required.

Evaluation of \ac{CAT} tools is an essential point that needs addressing. Currently, many studies draw conclusions simply based on speed and cognitive effort measured as words processed per unit of time \parencite{carl2015post}. I argue that not all effort is created equal. Translators carry out various kinds of tasks when translating, some which require skills aquired after years of practice and study. It is natural that translating a play on words, a complicated syntactical structure, an unclear passage in the \ac{ST}, etc. requires more effort and time than other tasks. These are precisely some of the tasks human translators should be carrying out and are trained to carry out. Offloading key tasks such as word and structure selection to an \ac{MT} engine and using qualified translators to ``clean up'' the output is a questionable way forward. As was discussed in \autoref{ch:cat}, this leads to negative attitudes towards \ac{PE}.

I propose a new approach to the development of \ac{CAT} tools that puts translators first from the outset. The methodology consists of the following steps:

\begin{enumerate}
\item \textit{Gather automation candidates:} theoretical work carried out in \ac{TS} and translators themselves through surveys are a valuable source of candidate taskes to be automated. They can provide hints as to which tasks within the translation process are the most tedious and can potentially be carried out without human intervention.
\item \textit{Select candidates addressable by \spacedlowsmallcaps{NLP}:} from the candidates gathered above, experts in computational linguistics can select the ones that can be resonably carried out with \spacedlowsmallcaps{NLP} tools. Other areas that plain software engineering can deal with could also be identified. 
\item \textit{Develop a tool or prototype:} design and implement a system to carry out the selected task. Emphasis is placed on prototype: at this stage, a full implementation may not be required as long as it can be reliably given to translators to use in the context of an experiment. Prototyping allows the tool development process to go much faster, quickly discarding bad ideas and gathering key feedback to improve the prototype once it is time to implement a full system. It could be argued that any difficulties in implementing a full system have not been caught. But in that case the design and efectiveness of the tool has been tested in a real setting and it justifies more resources being spent on overcoming the implementation difficulties. The opposite case, a good system that doesn't mix well with translators, is undesirable.
\item \textit{Intrinsic tool evaluation:} depending on the nature of the tool and whether it is a full tool or a prototype, an intrinsic evaluation should be carried out. At the very least, it should show acceptable performance in the experimental setup described in the next step.
\item \textit{Extrinsic evaluation:} an experimental setup is designed for translators to use the tool. Usage data is obtained by means of triangulation of data-gathering methods developed in \ac{TPR}. A good mix of methods (online and offline, verbal-report, product analysis and process monitoring) is preferable. The data gathered should then allow evaluation of the tool's effects on the whole translation process, including translators. The aim will be to answer the question ``Does this tool help translators carry out their work?''.
\end{enumerate}

The previous methodology should generate tools which do not force limitations onto translators and negatively affect their normal workflow. It is important to stress that the proposed methodology is designed for \ac{CAT} tool development. The idea of a prototype forces the researcher to not go down the rabbit whole of developing a perfect system. It instead forces them to take a step back and spend less time on the tool itself and more on how the tool is used and if it makes sense for translators. This approach is clearly not ideal for the development of a parsing algorithm, for example. \ac{CAT} is an eminently practical affair and the development of \ac{CAT} tools should reflect this fact.
