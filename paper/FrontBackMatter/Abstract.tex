% Abstract

\pdfbookmark[1]{Abstract}{Abstract} % Bookmark name visible in a PDF viewer

\begingroup
\let\clearpage\relax
\let\cleardoublepage\relax
\let\cleardoublepage\relax

\chapter*{Abstract} % Abstract name

\noindent This thesis identifies shortcomings with how \acf{CAT} tools are developed. Their final use as an aid to translators is often not fully considered and left to others to evaluate. A unified methodology is proposed which allows a \acs{CAT} tool to be evaluated intrinsically and extrinsically using methods that show the tool's effect on the whole translation process. Special emphasis is placed on prototyping as a resource-effective way to create tools and gain critical feedback before a full implementation. To evaluate the methodology's usefulness, StyleCheck is developed and evaluated using it. StyleCheck is a tool implemented in Grammatical Framework. It detects when a style guide rule is applied and gives a hint to the translator when it isn't. The use of methods developed in \acl{TPR} generates a wealth of data that gives detailed insights into how a tool performs. Results show StyleCheck is effective at getting a style guide to be applied, more than translating from scratch or post-editing, although more work on the user interface is required. The methodology is proven to be good at coming up with \acs{CAT} tool improvements, quickly protoyping them and evaluating them.

\endgroup			

\vfill
